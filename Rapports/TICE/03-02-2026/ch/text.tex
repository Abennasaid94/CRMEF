\chapter{Editeurs des Documents}
\label{doc}
\hypertarget{doc}{}
\minitoc
%\section*{Introduction}

%\section{Microsoft Office}
%\subsection{Informations d'Identification}
%\subsection{Informations Techniques}
%\subsection{Informations Fonctionnelles}

%\section*{Conclusion}
\section*{Introduction}
\section{Microsoft Office}
\subsection{Informations d'Identification}
Microsoft Office, aujourd'hui appelé Microsoft 365, est une suite de logiciels bureautiques développée par Microsoft Corporation. Elle a été créée en 1989 et fait l'objet de mises à jour régulières et continues, notamment dans sa version Microsoft 365. Il s'agit d'un logiciel propriétaire, principalement payant, proposé soit sous forme d'abonnement (Microsoft 365), soit sous forme de licence définitive (comme Office 2021 ou 2024). La suite est disponible dans de nombreuses langues, dont le français, l'anglais, l'espagnol et bien d'autres.
%\subsection{Informations Techniques}
%Sur le plan technique, Microsoft Office est compatible avec plusieurs systèmes d'exploitation, notamment Windows, macOS, Android et iOS, et peut également être utilisé directement via un navigateur web grâce à la version en ligne. La configuration minimale requiert un processeur d'environ 1,6 GHz, au moins 2 à 4 Go de mémoire RAM selon l'architecture du système, ainsi qu'un espace disque suffisant. Une connexion Internet est nécessaire pour l'activation, les mises à jour et l'utilisation des services en ligne. Le logiciel peut fonctionner hors ligne, en ligne, ou en mode hybride grâce à la synchronisation avec OneDrive.
%\subsection{Informations Fonctionnelles}
%D'un point de vue fonctionnel, l'objectif principal de Microsoft Office est de permettre la création, la modification et le partage de documents dans un cadre personnel, scolaire ou professionnel. La suite prend en charge différents types de contenus, tels que le texte avec Word, les données chiffrées et tableaux avec Excel, les présentations multimédias avec PowerPoint, ainsi que les courriels et agendas via Outlook. Elle permet également la gestion de notes, de bases de données, et la création de formulaires et de quiz, ce qui en fait un outil complet de productivité.

%\section{\LaTeX}
%\subsection{Informations d'Identification}
%\LaTeX{} est un logiciel de composition de documents basé sur le système \TeX{}, créé par \textbf{Leslie Lamport} en \textbf{1985}. Il est principalement utilisé pour la rédaction de documents scientifiques, techniques et académiques. \LaTeX{} est un logiciel \textbf{libre et open source}, distribué gratuitement. Il est maintenu et mis à jour régulièrement par une large communauté de développeurs. Le logiciel est disponible dans de nombreuses langues, notamment le français et l'anglais.
%
%\subsection{Informations Techniques}
%\LaTeX{} est compatible avec plusieurs \textbf{systèmes d'exploitation}, dont Windows, macOS et Linux. Il fonctionne à l'aide de distributions telles que \textbf{TeX Live}, \textbf{MiKTeX} ou \textbf{MacTeX}. La configuration minimale requise est relativement faible, nécessitant seulement un ordinateur standard avec quelques centaines de mégaoctets d'espace disque selon la distribution installée. \LaTeX{} peut être utilisé \textbf{hors ligne} après installation ou \textbf{en ligne} via des plateformes comme Overleaf.
%\subsection{Informations Fonctionnelles}
%L'objectif principal de \LaTeX{} est de permettre la \textbf{mise en forme professionnelle et automatisée de documents}. Contrairement aux logiciels WYSIWYG, l'utilisateur écrit le contenu sous forme de code, qui est ensuite compilé pour produire un document final. \LaTeX{} gère principalement le \textbf{texte}, les \textbf{formules mathématiques}, les \textbf{tableaux}, les \textbf{figures} et les \textbf{bibliographies}. Il est particulièrement apprécié pour la qualité de sa typographie et sa capacité à produire des documents longs et structurés.
\section*{Conclusion}